\chapter{Diskussion der Ergebnisse}
\label{evaluation}
%Die Beurteilung ist einer der wichtigsten Abschnitte der Arbeit
%- Sie enthält die Quintessenz des gesamten Projektes
%Viele lesen nur die Einführung und die Beurteilung an
%- Hier muss also alles Wichtige drin stehen!
%Hier beweisen Sie dass Sie …
%- die Aufgabe und deren Bedeutung verstanden haben
%- die Ergebnisse richtig zu interpretieren vermögen
%- wissen, worauf es bei diese Arbeit ankam

% - siehe auch Notizen in: https://docs.google.com/document/d/1jmJIAC5F2cos8MjODSyV-BzwO2jPDtREDRKhCnJRUts/edit#

Das nachfolgende Kapitel soll dazu dienen, dargestellte Ergebnisse kritisch zu evaluieren. Es soll eine generelle Auseinandersetzung mit der Methodik, möglichen Limitationen bei der Untersuchungen und den eigentlichen Ergebnissen erfolgen. Darüber hinaus werden mögliche Ansatzpunkte für weiterführende Arbeiten gegeben.

\textbf{Ergebnisse}

Nachfolgend soll eine kritische Betrachtung der vorgestellten Ergebnisse vorgenommen werden. Dafür soll zunächst untersucht werden, inwieweit die in \ref{introduction:fs} dargestellten Forschungsfragen beantwortet werden konnten. 

\hangindent+50pt \hangafter=1
\textbf{Forschungsfrage 1:}\textit{ Wie sehen allgemeine Veränderungsprozesse im Zuge der Digitalen Transformation aus?} Dies wurde in  \ref{problemfields:changepatterns} mittels einer Übersicht resultierend aus dem SLR 1 vorgenommen. Es hat sich gezeigt, dass es eine große Vielzahl an möglichen Ansatzpunkte für Veränderungen geben kann. Innerhalb der Auswertung hat sich gezeigt, dass eine Clusterung der Nennungen nötig war, da die Begrifflichkeiten teilweise stark voneinander abwichen, die Bedeutung sich jedoch stark ähnelte.  

\hangindent+50pt \hangafter=1
\textbf{Forschungsfrage 2:} \textit{Welche Probleme treten bei der Digitalen Transformation eines Großunternehmens auf?} Auch hierfür wurde eine Übersicht in \ref{problemfields:fields} vorgenommen. Wichtig für die Vergleichbarkeit mit den vorher erarbeiteten Veränderungsprozessmuster war die Gleichheit der zu untersuchenden Literatur in SLR 1. Dies konnte gewährleistet werden. Wiederrum war eine Clusterung der Ergebnisse notwendig.

\hangindent+50pt \hangafter=1
\textbf{Forschungsfrage 3:} \textit{Welche agilen Praktiken bzw. Methoden haben sich in großen Organisationen innerhalb des Transformationsprozesses etabliert?
} Hierfür diente die Durchführung eines zweiten SLR (vgl. \ref{agilepractices:extractions}). Mithilfe einer breiten Auswahl an Literatur, die durch ein bestimmtes Schema ausgewählt wurde, konnte eine große Reihe von agilen Methoden erarbeitet werden, die im Transformationsprozess eingesetzt werden. Wiederrum musste eine Clusterung vorgenommen werden, da teilweise Sonderformen von allgemeinen Methoden eingesetzt wurden. Teilweise war es schwer zu identifizieren, welcher Methode sich bestimmte Prozesse zuordnen lassen können. Unklare Aussagen mussten aussortiert werden, was das Ergebnis unter Umständen verfälscht haben könnte.

\hangindent+50pt \hangafter=1
\textbf{Forschungsfrage 4:} \textit{Wie können agile Praktiken bzw. Methoden dazu beitragen, die vorher erarbeiteten Probleme zu beheben?} Eine vollumfängliche Evaluation mit einem klaren Evaluationsschema und -ziel wurde durchgeführt (vgl. \ref{agilepractices:evaluation}). Hierfür konnten für fast alle agile Methoden genaue Aussagen für bestimmte Problemfelder identifiziert werden. Innerhalb der Evaluation wurden aus Platzgründen nicht alle in \ref{agilepractices:extractions} aufgeführten Methoden genutzt, sondern nach Wichtigkeit priorisiert, so dass einige aussortiert wurden. Somit hätten weiterführende Evaluationen der restlichen Methoden sicherlich zu einem erweiterten Ergebnis geführt. Die Anzahl der zu bearbeitenden Problemfelder (17 von 21) zeigt aber, dass sich anhand der gewählten agilen Methoden durchaus ein großes Problemspektrum, gerade im Hinblick auf interne Konflikte lösen lässt. Es hat sich gezeigt, dass für die Methode \textit{SAFe} kein positives Evaluationsergebnis erarbeiten lies. Hierfür könnte man durchaus das Auswahlkriterium in \ref{agilepractices:extractions} hinterfragen, in dem Sinne, als dass man eine inhaltliche Auswahl anstatt des Kriteriums der Nennungen hätte wählen können. Nachfolgende Untersuchungen könnten hier durchaus ansetzen, indem man die weiteren Methoden evaluiert. Auch hat die Evaluation ergeben, dass für bestimmte Problemfelde keine Aussagen gefunden werden konnten, obwohl dies durch die Definition der Methode durchaus zutreffend gewesen wäre. Dies ist sicherlich der Tatsache geschuldet, dass in keinen der Fallstudien die Problemfelder explizit untersucht wurden. Ein weiteres Kriterium, welches man hierbei beachten muss, dass die Anzahl der Aussagen mit der Anzahl der Nennungen korrelieren könnten. Beispielsweise ist die Chance auf mehr Aussagen bei der am meisten genannten Methode  \textit{Scrum} viel höher als bei anderen Methoden, da die Menge an verfügbaren Inhalt schlichtweg größer war. 

\hangindent+50pt \hangafter=1
\textbf{Forschungsfrage 5:} \textit{Welche Handlungsmuster lassen sich für einen erfolgreichen Einsatz agiler Praktiken bzw. Methoden im Transformationsprozess ableiten (Best Practices)?}  Hierfür wurden Leitlinien in \ref{agilepractices:bestpractice} aufgestellt. Diese wurden aus den vorhergehenden Ergebnissen abgeleitet, wodurch sie noch nicht als praxistauglich einzustufen sind. Es sind weitere Untersuchungen in diesem Feld nötig, um die erarbeiteten Leitlinien in der Digitalen Transformation zu validieren.

\clearpage

\textbf{Methodik und Limitationen} 

Als übergeordnete Methodik wurde ein zweiteiliges systematisches Literaturreview von Fallstudien (und weiterer Literatur mit Handlungsempfehlungen zur Thematik) vorgenommen. Dies diente dazu, um für beide Hauptteile eine Vielzahl an inhaltlichen Aussagen für die Untersuchungsziele zu generieren, sie zu clustern und zusammenzufassen. Diese Ergebnisse wurden für die Evaluation als Ausgangspunkt genutzt. 

Hinsichtlich der Literatursuche kann gesagt werden, dass die Menge an Ergebnissen in den verschiedenen Suchmaschinen stark variiert hat. Hierbei sollte bei nachfolgenden Untersuchungen in Betracht gezogen werden, die Suche auf bestimmte Suchmaschinen, beispielsweise \textit{Scopus}, einzugrenzen, da sie mehr Ergebnisse erzielten. Darüber hinaus können durchaus andere Suchmaschinen mit anderen, eventuell wirtschaftlicheren Schwerpunkten genutzt werden. Die gewählten Suchmaschinen hatten einen sehr technischen Bezug.

Auch kann die Wahl der Methodik an sich kritisch hinterfragt werden. Innerhalb der Evaluation hätten durchaus andere Formen der Untersuchung gewählt werden können, beispielsweise Beobachtungen, Experteninterviews oder Umfragen. Die Ergebnisse der vorgenommen Evaluation beruhen durchweg aus qualitativen Daten, da nur Aussagen aus der gefundenen Literatur genutzt wurden. Anhand qualitativer Methoden hätten gezieltere Ergebnisse erlangt werden können, da man hier spezifische Untersuchungen zu einzelnen Problemfeldern und agilen Methoden hätte durchführen können. Die Wahl auf das SLR mit abschließender Evaluation erfolgte deshalb, da man sich eine große Menge an bereits verfügbaren Daten versprach. Es wurden vornehmlich Fallstudien zu exakt der Thematik ausgewertet und für weitere Untersuchungen genutzt. Ziel war es, einen gegenwärtigen Status Quo über Problemfelder und Lösungsansätze der Digitalen Transformation zu schaffen und Anreize für nachfolgende Untersuchungen zu schaffen. Anschließend können spezifischere Untersuchungen stattfinden, die sich auf einzelne Problemfelder und agile Methoden beziehen. 

Innerhalb der Untersuchungen traf man auf vereinzelte Limitationen. Nicht alle Fallstudien nannten explizite Problemfelder und eingesetzte agile Methoden. Hierbei konnten nur klare Aussagen verwendet werden. Durch die Aktualität der Thematik Digitale Transformation wird die Menge an zu betrachteten Fallstudien zunehmend größer, so dass auch direkt nach der Untersuchung weitere potenzielle Literatur veröffentlicht wurde, die nicht Eingang in die Ergebnisse fand. Somit haben die vorgenommenen Untersuchungen keinen Anspruch auf Vollständigkeit. Mithilfe der Einschlusskriterien in beiden SLR wurde versucht,  die Auswahl einzugrenzen.

\textbf{Nachfolgende Arbeiten} 

Wie bereits aufgeführt, bilden die Ergebnisse Ansatzpunkte für nachfolgende Untersuchungen. Mithilfe von genauen Beobachtungen kann validiert werden, ob sich die Problemfelder tatsächlich mit den verlinkten agilen Methoden bearbeiten lassen (vgl. \ref{tab:clusteringfinal}). Des Weiteren sollten die Leitlinien mithilfe weiterer Untersuchungen validiert werden. Ein Kriterium, welches in der vorliegenden Arbeit nicht betrachtet wurde, sind mögliche kulturelle Unterschiede in einzelnen Unternehmen durch verschiedene Nationalitäten. In der Auswahl der Herkunft der Großunternehmen wurden keine Abgrenzungen vorgenommen. Sicherlich sollte untersucht werden, wie gerade bei der Konfliktlösung solche kulturellen Werte verschiedener Mitarbeitergruppen betrachtet werden können. Bestimmte agile Methoden, beispielsweise Design Thinking, bilden bereits erste Ansätze. Die Digitale Transformation bietet ein breites Feld für genaue Untersuchungen, gerade hinsichtlich der mit ihr einhergehenden Veränderungsprozesse.