\chapter{Verwandte Arbeiten}
\label{relatedwork}

% - Übersicht über artverwandte Arbeiten zum gleichen Thema
% - Wichtig: Herausarbeitung über thematische Einordnung der eigenen Leistung
% - + Darstellung über Mehrwert

Ein wesentlicher Schwerpunkt der vorliegenden Arbeit ist eine  Untersuchung von agilen Methoden als Einflussfaktor in der Digitalen Transformation. Durch die hohe Aktualität der Digitalen Transformation ist der Forschungsstand als relativ neuartig anzusehen. Trotzdem gibt es bereits eine Reihe veröffentlichter Publikationen, die sich mit einer ähnlichen Thematik beschäftigen. Diese sollen nachfolgend kurz vorgestellt und darüber hinaus der Beitrag der vorliegenden Arbeit hervorgehoben werden.

\shortciteA{vuksic_preliminary_2018} nahmen in ihrer Arbeit ein systematisches Literaturreview zur Digitalen Transformation vor. Sie stellen dar, dass der aktuelle Forschungsstand in diesem Bereich derzeit noch sehr unerforscht sei. Eins ihrer vordergründigen Erkenntnisse ist es, dass der Transformationsprozess sowohl das stetige Aufnehmen neuer Technologien in das eigene Geschäftsmodell, als auch interne organisationale Veränderungen bedingt (S. 1). Sie fanden heraus, dass neuartige Change Management Instrumente nötig seien, um den Herausforderungen der Digitalen Transformation zu begegnen.

Ein weiteres systematisches Literaturreview nahmen \shortciteA{dikert_challenges_2016} vor. Sie fassen detailliert zusammen, welche Herausforderungen und Erfolgsfaktoren in einer Agilen Transformation zu finden sind. Dabei wird ein starker Bezug auf die Anwendungen verschiedener agiler Methoden gelegt (S. 1).  Einen ähnlichen Ansatz verfolgen \shortciteA{osmundsen_digital_2018}. Methodisch wurde ebenfalls ein SLR gewählt, als Ergebnis werden Erfolgsfaktoren und Problemfelder der Digitalen Transformationen vorgestellt. Es wird vermehrt dargestellt, welchen Einfluss die Transformation auch auf die Organisation des Unternehmens hat (S. 1).

Einen direkten Bezug zwischen agilen Methoden und weitreichenden organisationalen Veränderungen beschriebt \citeA{hofert_agiler_2016} in ihrer Arbeit. Es wird deutlich, welche Möglichkeiten Führungskräfte aktuell haben, um Problemen für großangelegte Veränderungen zu begegnen, wie beispielsweise die Digitalisierung. Dafür werden eine Reihe agiler Methoden aufgeführt und erklärt, wie anhand dieser ein neuer Führungsstil implementiert werden kann.

\citeA{weinreich_lean_2016} legt den Kern seines Buches zu Lean Digitization ebenfalls auf eine Verbindung zwischen agilen Methoden und großen Veränderungen in Unternehmen. Es gibt einen klaren Bezug zur Digitalen Transformation. Es werden explizite Erfolgsfaktoren für den Einsatz agiler Methoden im Transformationsprozess genannt, beispielsweise für Design Thinking (S. 19f.). 

Die vorliegende Arbeit versucht ebenfalls eine Verbindung zwischen agilen Methoden und der Digitalen Transformation zu schaffen. Aufbauend auf den Einsatz eines mehrschichtigen SLR sollen explizite Veränderungsmuster und Problemfelder des Transformationsprozesses ermittelt werden. Daran anschließend werden agile Methoden hinsichtlich ihrer Einsetzbarkeit bei diesen Problemen untersucht. Der vordergründige Beitrag soll es sein, Muster für einen erfolgreichen Einsatz agiler Methoden zu finden. Inhaltlich kommt dies den Ergebnissen von \citeA{weinreich_lean_2016} nahe. Darüber hinaus verfolgt die vorliegende Arbeit einen sehr systematischen Einsatz, in dem eine ganze Reihe agiler Methoden untersucht werden. Es soll ein Überblick über Problemfelder und mögliche Lösungsansätze der Digitalen Transformation geschaffen werden, was Anreize für nachfolgende, spezifischere Untersuchungen schafft.
