\chapter{Einführung}
%- Hintergrund
%- Motivation
%- Ziele
%- Aufgaben
%- Allgemeine Beschreibung des Projektes
%- Worum geht es in dieser Arbeit?
%- Wer hat die Arbeit veranlasst und wozu?
%- Wer soll von den Ergebnissen profitieren?
%- Welches Problem soll gelöst werden? Warum?
%- Unter welchen Umständen braucht man eine Verbesserung?
%- Was ist der Stand der Technik?
%- Welche noch offenen Probleme gibt es?
%- Worin unterscheidet sich mein Ansatz von den bisherigen?
%- Welche Ziele hat die Arbeit?
%- Wie will ich diese Ziele erreichen?
%- Was habe ich im Einzelnen vor?


\todots

\section{Problemstellung}

% - grundlegende Einleitung (siehe Exposé!)
\todots

\section{Forschungsfragen und Zielstellung}

% - Darstellung der Forschungsfragen (siehe Exposé!)

\todots

\section{Aufbau der Arbeit}

% - modularer Aufbau, zu Erstellen: Grafik, woraus ersichtlich wird, wie die einzelnen Kapitel zusammenhängen

\todots

\section{Methodisches Vorgehen}

% - allgemeines Vorgehen, mit Verweis auf die methodischen Unterkapitel der einzelnen Abschnitte. Systematischer Literaturreview + Metastudie als spezielle Form, Evaluation am Ende der zweiten Metastudie

\todots
