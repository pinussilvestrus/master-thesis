\chapter{Einführung}
%- Hintergrund
%- Motivation
%- Ziele
%- Aufgaben
%- Allgemeine Beschreibung des Projektes
%- Worum geht es in dieser Arbeit?
%- Wer hat die Arbeit veranlasst und wozu?
%- Wer soll von den Ergebnissen profitieren?
%- Welches Problem soll gelöst werden? Warum?
%- Unter welchen Umständen braucht man eine Verbesserung?
%- Was ist der Stand der Technik?
%- Welche noch offenen Probleme gibt es?
%- Worin unterscheidet sich mein Ansatz von den bisherigen?
%- Welche Ziele hat die Arbeit?
%- Wie will ich diese Ziele erreichen?
%- Was habe ich im Einzelnen vor?

\todo{Umformulierung?}
Die gegenwärtig stark einhergehende Digitalisierung des privaten, beruflichen und öffentlichen Lebens verändert die Art wie Unternehmen untereinander konkurrieren, Werte schaffen und mit ihren Geschäftspartnern und Kunden interagieren \cite[S. 1]{oswald_digitale_2018}.

Die sogenannte Digitale Transformation wird ein zunehmend wichtig werdender Veränderungsprozess, um die gegenwärtigen Potenziale neuer Innovationen wie Big Data, künstliche Intelligenz oder Cloud Computing konsequent auszunutzen und stetig Wettbewerbsvorteile zu generieren \cite[S. 2]{oswald_digitale_2018}. Klassische Führungskonzepte greifen nicht mehr, Stichworte wie Flexibilität, Schnelligkeit, Dynamik und Kundenorientierung sind essentielle Voraussetzungen. 

Das Aufbrechen alteingesessener Strukturen bis hin zur Transformation zu einem digitalen Unternehmen bringt jedoch hohe Probleme mit sich. Bei einer erfolgreichen Umsetzung des Veränderungsprozesses spielen eine große Reihe von Einflüssen mit ein. Oft bremsen gerade alte Strukturen den Erfolg der Veränderungen, was gerade bei der Forderung nach Flexibilität und Dynamik ein großes Problem darstellt
\cite[S. 196]{appelfeller_digitale_2018}


Sogenannte agile Methoden können helfen, Probleme des klassischen Change Managements in der Digitalen Transformation zu bearbeiten. Praktiken wie z.B. Design-Thinking oder DevOps haben sich bereits als Projektmanagement-Instrumente bewährt \cite[S. 7]{deeken_agiles_2018}. Interessant ist jedoch auch die Herangehensweise, sie im größeren Umfeld einer Organisationskultur einzusetzen, um mithilfe der Bildung einer Agilen Kultur den Transformationsprozess zu optimieren \cite[S. 140]{hofert_agiler_2016}.


\section{Problemstellung}

% - grundlegende Einleitung (siehe Exposé!)
\todots

\section{Forschungsfragen und Zielstellung}

% - Darstellung der Forschungsfragen (siehe Exposé!)

\todots

\section{Aufbau der Arbeit}

% - modularer Aufbau, zu Erstellen: Grafik, woraus ersichtlich wird, wie die einzelnen Kapitel zusammenhängen

\todots

\section{Methodisches Vorgehen}

% - allgemeines Vorgehen, mit Verweis auf die methodischen Unterkapitel der einzelnen Abschnitte. Systematischer Literaturreview + Metastudie als spezielle Form, Evaluation am Ende der zweiten Metastudie

\todots
