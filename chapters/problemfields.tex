\chapter{Problemfelder innerhalb des Change Managements}
\label{problemfields}

% - erster Hauptteil der Arbeit
% - Herausarbeitung von Veränderungsprozessen und Problemfelder innerhalb der DT

Nach den theoretischen Ausführungen der vorangestellten Kapiteln soll nun inhaltlich auf den ersten Hauptschwerpunkt der vorliegenden Arbeit eingeführt werden. Um in \ref{agilepractices} eine Reihe agiler Methoden hinsichtlich ihrer Anwendbarkeit in Problemfeldern der Digitalen Transformation zu evaluieren, müssen letztere zunächst erarbeitet werden. Dafür soll ein systematisches Literaturreview (\gls{SLR}) vorgenommen werden. Aus diesem werden  zunächst Veränderungsprozessmuster innerhalb des Transformationsprozesses  zusammengefasst.  Zusätzlich werden auf  Grundlage dessen eine Reihe von problematischen Schwerpunkten erarbeitet. Nachfolgend soll zunächst das methodische Vorgehen im aktuellen Kapitel geschildert werden. Es schließt sich eine gesonderte Übersicht der erschlossenen Literatur an, auf dessen Grundlage  die Ergebnisse aufgeführt werden.  

\section{Methodisches Vorgehen}
\label{problemfields:methods}

% - genaue Darstellung über Vorgehen in der SLR + Metastudie
% - Suchmethodik, Keywords, Kriterien, Auswahlprozes ...
% - https://docs.google.com/document/d/1wzYRearLcVTlKJxPp6Mz5U860VoXzkNX7armGeCoxt4/edit

\todots

\section{Literaturübersicht}

% - Fallstudien
% - tabellarische Darstellung (https://docs.google.com/document/d/1caHZ-pLGa\_L-TfO4nOh2zZLdAa5nazKaoUZDeDmN90M/edit)
% - Inhaltsangabe jeder Arbeit (Zusammenfassung)

\todo{Siehe Tabelle A1 und A2 im Anhang}

\todots

\section{Veränderungsprozessmuster innerhalb der Digitalen Transformation}
\label{problemfields:changepatterns}

% - tabellarische Kreuzmatrix (siehe Link oben)
\todo{Siehe Tabelle A3 im Anhang}

\begin{table}[ht]
	\centering
	\caption{Auswertung Clustering Veränderungsprozessmuster (kurz)}
	\begin{tabular}{|c|c|}
		\hline
		\textbf{Veränderungsprozessmuster}& \textbf{Anzahl Nennungen} \\
		\hline
		Einführung einer Multi-Kanal-Strategie   & 6  \\
		Wechsel vom Offline- zum Online-Vertrieb & 5  \\
		Erschaffung neuer digitaler Produkte     & 12 \\
		Tendenz zur Kundenorientierung           & 16 \\
		Digitalisierung des Geschäftsmodell      & 11 \\
		Digitaler Wissensaufbau im Unternehmen   & 12 \\
		Datengestützte Verbesserungsprozesse     & 5  \\
		Erzeugung eines digitalen Ökosystems     & 7  \\
		Digitalisierung von internen Prozessen   & 7  \\
		Einbindung innovativer Technologien      & 6  \\
		Aufbau technisches Sicherheitskonzept    & 4  \\
		Digitale Neuausrichtung der Organisation & 13 \\
		Innovationsförderung                     & 7  \\
		Kooperation mit externen Treibern        & 5 \\
		\hline
	\end{tabular}
	\label{tab:clusteringvpshort}
\end{table}

\todots

\section{Identifikation von Problemfeldern}

% - tabellarische Kreuzmatrix (siehe Link oben)
\todo{Siehe Tabelle A4 im Anhang}

\begin{table}[ht]
	\centering
	\caption{Auswertung Clustering Problemfelder (kurz)}
	\begin{tabular}{|c|c|}
		\hline
		\textbf{Problemfeld}& \textbf{Anzahl Nennungen} \\
		\hline
		Unternehmensweite Kommunikationsprobleme        & 5  \\
		Festhalten an verfestigten Strukturen           & 7  \\
		Zeit - und Marktdruck                           & 5  \\
		Unterschätzung der Komplexität                  & 4  \\
		Fehlende Kontinuierliche Verbesserungsprozesse  & 7  \\
		Störung durch oberes Management (Top-Down)      & 7  \\
		Konflikte zwischen IT und Business              & 4  \\
		Fehlende frühe Einbeziehung aller Mitarbeiter   & 3  \\
		Sicherheitsprobleme                             & 2  \\
		Fehlende Kundenorientierung                     & 13 \\
		Fehlendes technischen Know-How                  & 10 \\
		Fehlende monetäre Ressourcen                    & 2  \\
		Rechtliche Bestimmungen und Datenschutz         & 5  \\
		Fehlende Transparenz (intern und extern)        & 4  \\
		Fehlende Partnerschaften                        & 6  \\
		Fehlende Innovationskultur                      & 6  \\
		Langsame Entscheidungsprozesse                  & 3  \\
		Unternehmenskulturelle Probleme                 & 8  \\
		Fehlendes Vertrauen, Akzeptanz und Bereitschaft & 9  \\
		Unklare Verantwortlichkeiten                    & 5  \\
		Fehlende digitale Strategie                     & 6 \\
		\hline
	\end{tabular}
	\label{tab:clusteringpfshort}
\end{table}

\todots

\section{Zusammenfassung}


