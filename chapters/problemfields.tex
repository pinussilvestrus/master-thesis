\chapter{Problemfelder innerhalb des Change Managements}
\label{problemfields}

% - erster Hauptteil der Arbeit
% - Herausarbeitung von Veränderungsprozessen und Problemfelder innerhalb der DT

Nach den theoretischen Ausführungen der vorangestellten Kapiteln soll nun inhaltlich auf den ersten Hauptschwerpunkt der vorliegenden Arbeit eingeführt werden. Um in \ref{agilepractices} eine Reihe agiler Methoden hinsichtlich ihrer Anwendbarkeit in Problemfeldern der Digitalen Transformation zu evaluieren, müssen letztere zunächst erarbeitet werden. Dafür soll ein systematisches  Literaturreview vorgenommen werden. Aus diesem werden  zunächst Veränderungsprozessmuster innerhalb des Transformationsprozesses  zusammengefasst.  Zusätzlich werden auf  Grundlage dessen eine Reihe von problematischen Schwerpunkten erarbeitet. Nachfolgend soll zunächst das methodische Vorgehen im aktuellen Kapitel geschildert werden. Es schließt sich eine gesonderte Übersicht der erschlossenen Literatur an, auf dessen Grundlage  die Ergebnisse aufgeführt werden.  

\section{Methodisches Vorgehen}
\label{problemfields:methods}

% - genaue Darstellung über Vorgehen in der SLR + Metastudie
% - Suchmethodik, Keywords, Kriterien, Auswahlprozes ...
% - https://docs.google.com/document/d/1wzYRearLcVTlKJxPp6Mz5U860VoXzkNX7armGeCoxt4/edit

\todots

\section{Literaturübersicht}

% - Fallstudien
% - tabellarische Darstellung (https://docs.google.com/document/d/1caHZ-pLGa\_L-TfO4nOh2zZLdAa5nazKaoUZDeDmN90M/edit)
% - Inhaltsangabe jeder Arbeit (Zusammenfassung)

\todots

\section{Veränderungsprozessmuster innerhalb der Digitalen Transformation}
\label{problemfields:changepatterns}

% - tabellarische Kreuzmatrix (siehe Link oben)

\todots

\section{Identifikation von Problemfeldern}

% - tabellarische Kreuzmatrix (siehe Link oben)

\todots

\section{Zusammenfassung}


