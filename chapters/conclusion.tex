\chapter{Zusammenfassung}
\label{conclusion}

Das übergeordnete Ziel der vorliegenden Arbeit sollte die Evaluation verschiedener agiler Methoden als mögliche Change Management Instrumente innerhalb der Digitalen Transformation von Großunternehmen sein. Es wurden beispielhafte Leitlinien herausgearbeitet, die zeigen, in welcher Art und Weise Agilität im Allgemeinen und agile Werte im Besonderen in Veränderungsprozessen eingesetzt werden können.

Als erster Schwerpunkt konnten eine Reihe von Veränderungsprozessmuster in der Digitalen Transformation mittels eines systematischen Literaturreviews erarbeitet werden (vgl. \ref{problemfields}). Diese zeigen, in welchen Bereichen Veränderungen entstehen können. Darüber hinaus wurden insgesamt 21 Problemfelder der Digitalen Transformation identifiziert. Diese zeigen durch die Analyse von mehreren Fallstudien einen aktuellen Status Quo, in welchen Bereichen es vermehrt zu Problemen in den Veränderungsprozessen kommen kann. Beispielsweise hat sich gezeigt, dass es gegenwärtig starke Probleme in der Kundenorientierung der Produktentwicklung gibt. Teilweise fehlte diese komplett.

Die erarbeiteten Problemfelder bildeten die Grundlage für die nachfolgende Evaluation agiler Methoden im Kontext der Digitalen Transformation (vgl.  \ref{agilepractices}). Dafür wurde zunächst ein weiteres SLR durchgeführt, um eine Reihe agiler Methoden zu extrahieren, die vermehrt im Transformationsprozess eingesetzt werden. Anschließend wurden in einer systematischen Evaluation neun ausgewählte Methoden hinsichtlich ihrer Anwendbarkeit in bestimmten Problemfeldern untersucht. Es hat sich gezeigt, dass die Ansatzpunkte sehr unterschiedlich und durchaus vielversprechend sind. Beispielsweise hat sich Scrum als sehr effektiv gezeigt, in dem es durch seine agilen Werte in vielen Problemfeldern eingesetzt werden kann. Es hat sich verdeutlicht, dass vor allem durch die Mischung verschiedener agiler Ansätze große Erfolgschancen erreichbar sind.

Die erarbeiteten Leitlinien können als Ausgangspunkt für erfolgreiche Implementationen agiler Methodiken in der Digitalen Transformation genutzt werden. Es hat sich gezeigt, dass die Ergebnisse einen sehr allgemeinen Überblick über den Einsatz verschiedener agiler Methoden bieten. Nachfolgende Untersuchungen haben die Chance, agile Methoden in speziellen Problemszenarien zu beobachten. So kann gezeigt werden, inwieweit die Ergebnisse, die zunächst auf spezifische Aussagen aus bereits vorhandenen Fallstudien basieren, in der Praxis umgesetzt werden können. 

Die Digitale Transformation ist ein Prozess, der aktuell eine sehr hohe Bedeutung genießt, und dessen noch eher junger Forschungsstand ein großen Raum für neue Erkenntnisse bietet. Die vorliegende Arbeit hat unter anderem auch gezeigt, dass der Prozess mit vielen Problemen behaftet sein kann, er aber dennoch in der Agenda eines jeden Großunternehmens verankert sein sollte, um mit neuen, digitalen Produktangeboten konkurrenzfähig zu bleiben.