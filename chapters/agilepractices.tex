\chapter{Etablierung Agiler Praktiken}
\label{agilepractices}

% - erkenntnisreichster Teil der Arbeit!
% - Erarbeitung von verwendeten agilen Praktiken
% - Evaluation hinsichtlich ihrer Anwendbarkeit bei den vorher erarbeiteten Problemfeldern

\todots

\section{Methodisches Vorgehen}
\label{agilepractices:methods}

% - genaue Darstellung über Vorgehen in der SLR + Metastudie: 
% - Suchmethodik, Keywords, Kriterien, Auswahlprozes ...
% + Schema der Evaluation (Problemfelder)
% https://docs.google.com/document/d/1j8hegu3FgQk2fPMVjo8IUPWIXwNjlZjT42rHxg7KZ1I/edit

\todots

\section{Literaturübersicht}

% - tabellarische Übersicht, Fallstudien + allgemeine Sekundäre Literatur
% - Inhaltsangabe jeder Arbeit (Zusammenfassung)

\todo{siehe Tabelle 5-7 im Anhang}

\todots

\section{Einsatz agiler Praktiken im Unternehmen}

% - Ergebnisse der SLR - Auflistung der genutzten Methoden

\todots

\section{Darstellung und Evaluation verschiedener agiler Methoden}

% - systematisch: erst Definition der Methode, dann Evaluation im Bezug auf die Problemfelder
% - siehe Key findings: https://docs.google.com/document/d/1QayCknP1SgspsPvXlU_FR5u0h2fNzhfQshNOrAuTiB8/edit

\todots

\subsection{Scrum}

\todots

\subsection{Design Thinking}

\todots

\subsection{DevOps}

\todots

\subsection{Kanban}

\todots

\subsection{XP}

\todots

\subsection{SAFe}

\todots

\subsection{Digital Innovation Lab}

\todots

\subsection{Minimum Valuable Product}

\todots

\subsection{Squads and Tribes}

\todots


\section{Formulierung agiler Best-Practice Szenarien}

% - Erstellung von Thesen für Best-Practice Szenarien (im Bezug auf allen Ergebnissen von vorher, vor allen auch 4.3)

\todots


