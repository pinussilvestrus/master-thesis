\chapter{Theoretische Grundlagen}
\label{background}

%- Allgemeine Wissensgrundlagen des Fachgebiets
%- Spezielle Grundlagen, die für das Verständnis erforderlich sind
%- Rahmenbedingungen für die Arbeit
%- Ausführungen zum Stand des Wissens / der Technik
%Als Leitprinzip gilt: Nur Informationen erwähnen, die
%- später benötigt werden,
%- notwendig sind, um die Arbeit oder ihre Motivation zu verstehen
%Das heißt insbesondere,
%- keine Inhalte aus Lehrbüchern, außer
%- diese werden benötigt, um Problemstellung oder Lösungsweg zu definieren.

\todots

\section{Digitale Transformation}

\todots

\section{Change Management}

\todots

\section{Agilität}

\todots

\section{Agiles Management}

\todots

\section{Abgrenzung Großunternehmen}