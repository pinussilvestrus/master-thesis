\ifgerman{\chapter*{Inhaltsangabe}}{\chapter*{Abstract}}

Dies Ziel einer jeden Digitalen Transformation ist eine Neuausrichtung von Unternehmen, um den Herausforderungen und Ansprüchen der gegenwärtigen Epoche der Digitalisierung gewachsen zu sein. Bei diesen großangelegten Veränderungen kann es gleichermaßen zu weiteren Herausforderungen und Problemen kommen, die mithilfe bestimmter Techniken begegnet werden können. Eine erfolgreiche Digitale Transformation ist notwendig, um dauerhaft konkurrenzfähig zu bleiben. 

Die vorliegende Arbeit untersucht mithilfe eines zweiteiligen systematischen Literaturreviews mögliche Problemfelder der Digitalen Transformation von Großunternehmen. Des weiteren werden agile Methoden erarbeitet, die vermehrt Eingang in den Transformationsprozess gefunden haben. In einer anschließenden Evaluation werden beide Untersuchungsergebnisse miteinander verknüpft, in dem weiter untersucht wird, in welchem Problembereichen die ausgewählten agilen Methoden einsetzbar sind.

Insgesamt wurden im ersten Review 23 Fallstudien systematisch untersucht, zusammengefasst und geclustert, wodurch sich 21 Problemfelder der Digitalen Transformation erarbeiten ließen. Im zweiten Review wurden 32 Fallstudien und Fachliteratur zu der Thematik untersucht, wodurch 18 agile Methoden extrahiert werden konnten. Diese wurden anschließend mittels eines klaren Schemas evaluiert.

Als Ergebnis werden eine Reihe von Leitlinien für eine erfolgreiche Anwendung von agilen Methoden in der Digitalen Transformation aufgestellt. Diese können als Ansatzpunkt für die Planung von Veränderungsprozessen im Kontext der Digitalen Transformation genutzt werden.
